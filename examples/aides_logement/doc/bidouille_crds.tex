\documentclass[french,11pt]{article}

\usepackage[T1]{fontenc}
\usepackage[utf8]{inputenc}
\usepackage{fullpage}
\usepackage{babel}
\usepackage{amsmath}
\usepackage{amssymb}
\usepackage{eurosym}

\title{Invariant caché CRDS et APL}
\author{Lilya \bsc{Slimani} \& Denis \bsc{Merigoux}}
\date{21 avril 2022}

\begin{document}
\maketitle

Ce document a pour but d'expliciter un invariant caché régissant les étapes finales
du calcul de l'APL (secteur locatif, logement-foyer et accession à la propriété).

Les articles D823-10, D823-16 et D823-14 du code de la construction et de l'habitation
définissent tous la pénultième étape du calcul de l'APL par la phrase suivante:

\begin{quote}
  \itshape Ce dernier résultat, obtenu par application des dispositions précédentes, est diminué d'un montant
  représentatif des contributions sociales qui s'y appliquent, arrondi à l'euro inférieur,
  puis majoré de ce montant représentatif.
\end{quote}

La formulation est sybilline, il nous faut l'expliciter. En effet, l'APL est soumise
à la Contribution de la Réduction pour la Dette Sociale (CRDS) prévue par l'ordonnance
96-50 du 24 janvier 1996 au titre de son article 14. Par contre, l'APL
est exonérée de la Contribution Sociale Généralisée (CSG) au titre de l'article
L136-1-2 du code de la sécurité sociale.

Ainsi, ce qui devra être distribué au ménage n'est pas le montant de l'APL tel que
défini par le code de la construction et de l'habitation, mais le montant minoré
de la CRDS prélevée à la source. Or, il semblerait que par la formulation des articles
D823-10, D823-16 et D823-14 du code de la construction et de l'habitation, l'administration
a voulu exercer un contrôle fin du montant effectivement versé au ménage: il faut
arrondir le montant à l'euro inférieur.

Afin d'obtenir le résultat souhaité en bout de chaîne après prélèvement de la CRDS,
les rédacteurs du code de la construction et de l'habitation effectuent donc cette
manipulation peu orthodoxe qui consiste à anticiper le prélèvement de la CRDS,
l'arrondir à l'euro inférieur, puis rajouter la CRDS pour qu'elle soit prélevée
conformément à l'ordonnance de 1996. Cela donne donc les formules suivantes,
avec $\textrm{APL}_{\text{brut}}$ et $\textrm{APL}_\textrm{net}$ les montants
d'APL avant et après prélèvement de la CRDS, et $\textrm{APL}_0$ le montant
initial de l'APL avant la rectification de l'arrondi:

\begin{align*}
  \textrm{APL}_{\textrm{brut}} & = \lfloor \textrm{APL}_0 - \textrm{CRDS}(\textrm{APL}_0)\rfloor + \textrm{CRDS}(\textrm{APL}_0) \\
  \textrm{APL}_{\textrm{net}}  & = \textrm{APL}_{\textrm{brut}}  - \textrm{CRDS}(\textrm{APL}_{\textrm{brut}} )                  \\
                               & = \lfloor \textrm{APL}_0 - \textrm{CRDS}(\textrm{APL}_0)\rfloor
\end{align*}

Cependant, les plus attentifs d'entre vous remarqueront que cette dernière simplication
des termes de CRDS n'est pas exacte! En effet, il se cache en fait un terme

\begin{align*}
  \Delta & =  \textrm{CRDS}(\textrm{APL}_0) - \textrm{CRDS}(\textrm{APL}_{\textrm{brut}} )                                                                                                 \\
         & = \textrm{CRDS}(\textrm{APL}_0 -\textrm{APL}_{\textrm{brut}}  )                                                                   \quad \text{ car la CRDS est proportionnelle} \\
         & = \textrm{CRDS}( \textrm{APL}_0 - \textrm{CRDS}(\textrm{APL}_0) - \lfloor \textrm{APL}_0 - \textrm{CRDS}(\textrm{APL}_0) \rfloor)
\end{align*}

En notant $\textrm{APL}_{\textrm{non arrondie}} \triangleq  \textrm{APL}_0 - \textrm{CRDS}(\textrm{APL}_0)$, alors on a

\[
  \Delta = \textrm{CRDS}(\textrm{APL}_{\textrm{non arrondie}} - \lfloor \textrm{APL}_{\textrm{non arrondie}} \rfloor)
\]

Dès lors, il est facile de constater que $0 \; \text{\euro} \leqslant \textrm{APL}_{\textrm{non arrondie}} - \lfloor \textrm{APL}_{\textrm{non arrondie}}\rfloor < 1 \; \text{\euro}$.
Or, le taux de la CRDS étant de 0,5\%, on a forcément $\Delta < 0,01\;\text{\euro}$ et
donc pour la pratique du versement $\Delta = 0$, ce qui explique pourquoi on a

\[
  \textrm{APL}_{\textrm{net}} = \lfloor \textrm{APL}_{\textrm{non arrondie}}\rfloor
\]

À noter cependant que cette astuce ne marche que parce que le taux de CRDS est
inférieur à 1\% !

\end{document}
